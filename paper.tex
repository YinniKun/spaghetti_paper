
% Title page
\date{}
\title{SPAGHETTI Optimizes the Quantitative Analysis of \textit{In Vitro} Phase Contrast Miscroscopy through a Generative Deep Learning Model}
\author[1-3]{Zhi Fei (Richard) Dong}
\author[1-3]{Chris McIntosh}
\author[1-3]{Gregory W. Schwartz}
\affil[1]{Princess Margaret Cancer Centre, University Health Network, Toronto, ON M5G 1L7, Canada}
\affil[2]{Department of Medical Biophysics, University of Toronto, Toronto, ON M5G 1L7, Canada}
\affil[3]{Vector Institute, Toronto, ON M5G 1M1, Canada}
\affil[
]{\href{mailto:gregory.schwartz@uhn.ca}{email:gregory.schwartz@uhn.ca}}
%\affil[*]{These authors contributed equally to this work}

\maketitle

\begin{refsegment}

\begin{abstract}
  \noindent \textbf{Motivation:} This is important.

  \noindent \textbf{Results:} We present this.

  \noindent \textbf{Availability:} \todo{transfer and clean up code + add to SchwartzLab Github} is freely available at \url{}.
\end{abstract}

\section*{Introduction}

Order of the whole paper: Abstract, Introduction, Results, Discussion, Methods,
Data Availability, Code Availability, Acknowledgments, Funding, Author
Contributions, Competing Interests, Supplementary Information

There is where the intro
goes~\autocite{schwartzTooManyCellsIdentifiesVisualizes2020}. This includes a
short literature review highlighting the gaps that your paper fills in. Ends
with a summary of the paper.

\section*{Results}

\section*{Directory info}

For the code, we want \texttt{main.tex} to have everything (paper and rebuttal).
We comment out rebuttal for initial submission. Paper text is in
\texttt{paper.tex} and rebuttal text is in \texttt{rebuttal/rebuttal.tex}.
Figures go in the \texttt{figures} folder (both \texttt{pdf} and \texttt{svg}),
rebuttal figures go in \texttt{rebuttal/figures}.

To build the document, run \texttt{latexmk -lualatex main.pdf}.

\subsection*{Figure references}

Let me tell you about this amazing result
(\Cref{fig:amazing,fig:astounding,fig:incredible,fig:anotherAmazing,fig:anotherWow,sfig:okay,sfig:maybe,sfig:alright,tab:description,stab:goodStuff}).

\subsection*{Revisions}

\qc{fixed}{\highLight{We also did stuff here for the revision that will appear
    in the rebuttal. Notice how references to figures are correct
    (\Cref{fig:astounding}).}}

\subsection*{Figure style}

\begin{enumerate}
\item \textit{Make sure your figure style is consistent throughout!}
\item We usually use 1px line width for plots (or 0.2mm, but Scientific Inkscape
  defaults to 1px so it's easy to just use that).
\item Nothing smaller than 8pt font, if you can't fit the text with that size in
  the plot then make the plot bigger. For panel labels (a, b, etc.) use 12pt
  bold lowercase.
\item Consistent color scheme throughout. Don't mix color schemes. We usually
  use Set1 from ColorBrewer.
\item Use alignment, don't eyeball centering. Make sure plots and panel labels
  are aligned as much as possible with each other.
\item Don't use gridlines unless they are absolutely necessary
\item Don't squish plots. It looks bad. If you have trouble resizing with too
  much text, use \url{https://github.com/burghoff/Scientific-Inkscape}.
  Scientific Inkscape has a \textit{lot} of very useful utilities for editing
  scientific figures.
\end{enumerate}

\subsection*{Language}

Importantly, in this section you should not write dryly. You must tell a story
with a high-level view of the methods, more detail goes into the methods
section. How specifically should this be written? Read other papers to see how
they do it! Check out \textit{Nature Methods} papers. For example from us
(figures don't match, just using as an example):

\begin{quote}
In order to understand how survival programs could be affected by the duration
of treatment, we sought to characterize the unique expression profiles among
persister-cell populations. We identified differentially expressed genes between
control and persister cells of each cell line separately and aggregated the
complete list of genes using rank product analysis
\autocite{breitlingFEBSLetters2004}. The batch functionality of TMCI allowed us
to efficiently visualize the distribution of top-ranking most differentially
expressed genes across the entire data-set collection. From these
visualizations, we identified \textit{ID2} as one of the most highly upregulated
genes across long-term treated cells in comparison to controls (rank product: 4,
permutation test: \(p < \SI{2.22e-16}{}\)), but not among the short-term treated
cells (rank product: 380, permutation test: \(p < \SI{2.22e-16}{}\))
(\Cref{fig:astounding,stab:goodStuff,sfig:okay}). Comparison of
  \textit{ID2} expression between each control and corresponding treatment arm
  showed a significant increase of \(\log_2\) fold change values for all cell
  lines (Mann-Whitney \(U\) test: \(p < \SI{0.05}{}\)), regardless of treatment
  duration, with the exception of short-term treated DND-41 cells
  (\Cref{fig:amazing} and \Cref{stab:goodStuff}). \textit{ID2} is known to play
  a role in tumorigenesis as a key regulator of cell-cycle progression and
  overexpression of \textit{ID2} in cell-line experiments modulates
  proliferative capacity and cell invasiveness \autocite{itahana_role_2003,
    stighall_high_2005}. Differential \textit{ID2} expression in our analysis
  suggests varying proliferative activity between treatment durations.
\end{quote}

\section*{Discussion}

This section is for summarizing the paper, offering more speculative
interpretations, and suggesting future work and impact.

\section*{Methods}

\subsection*{We did stuff}

This is how we did it. We used \SI{15}{\micro\meter} keyboards.

\section*{Data Availability}

We obtained spatial transcriptomic data of human lymph node from
\url{https://www.10xgenomics.com/datasets/human-lymph-node-1-standard-1-0-0},
mouse hypothalamic preoptic region from
\url{https://datadryad.org/stash/dataset/doi:10.5061/dryad.8t8s248}, LUAD from
the Gene Expression Omnibus under accession number GSE189487, and human
colorectal cancer (Visium HD) from
\url{https://www.10xgenomics.com/datasets/visium-hd-cytassist-gene-expression-libraries-of-human-crc}.

\section*{Code availability}

\todo{XXXXXX} is available at
\url{https://github.com/schwartzlab-methods/XXXXXXXX} or as a Singularity image
at \url{https://cloud.sylabs.io/library/you/collection/XXXXXXXXX.sif} with a
tutorial at \url{https://github.com/schwartzlab-methods/XXXXXX#vignette}.
Scripts for generating the figures and plots of the manuscript can be found at
\url{https://github.com/schwartzlab-methods/XXXXXXX_paper_figures}.

\section*{Acknowledgments}

We thank a lot of people.

\section*{Funding}

This work was supported by the University of Toronto Data Sciences Institute
Research Software Development Support Program (G. W. S.), the Canadian Cancer
Society Challenge Grant (grant 707484; G. W. S.), the Natural Sciences and
Engineering Research Council of Canada (grants RGPIN-2023-04713 and
DGECR-2023-00395; G. W. S.), the Social Sciences and Humanities Research Council
(grant NFRFE-2022-00681; G. W. S.), the Canada Research Chairs Program (G. W.
S.), and the Princess Margaret Cancer Foundation (G. W. S.).

\section*{Authors Contributions}

G. W. S. conceived and supervised the project. Y. O. U. did everything else.

\section*{Competing Interests}

G. W. S. conceived and supervised the project. Y. O. U. developed the XXXXXXXX
method, software, and benchmarks. Y. O. U. ran and analyzed benchmarks. Y. O. U.
generated experimental results. Y. O. U. ran and analyzed data. Y. O. U. and G.
W. S. wrote and edited the manuscript. All authors reviewed the manuscript.

\clearpage

\end{refsegment}
\printbibliography[segment=1]

\clearpage

\begin{figure}[htbp]
\centering
\includepdf[fitpaper=true,pages=-]{figures/figure_LUAD_CRC.pdf}
\end{figure}

\clearpage

\begin{figure}[htbp]
  {\phantomsubcaption\label{fig:amazing}}
  {\phantomsubcaption\label{fig:astounding}}
  {\phantomsubcaption\label{fig:incredible}}
  \caption{\panel{sub@fig:amazing,sub@fig:astounding,sub@fig:incredible}, Many cool things.
    \panel{sub@fig:amazing}, This is amazing. \panel{sub@fig:astounding}, This
    is astounding. \panel{sub@fig:incredible}, This is incredible.
  \label{fig:incredibleOne}}
\end{figure}

\clearpage

\begin{figure}[htbp]
\centering
\includepdf[fitpaper=true,pages=-]{figures/figure_LUAD_CRC.pdf}
\end{figure}

\clearpage

\begin{figure}[htbp]
  {\phantomsubcaption\label{fig:anotherAmazing}}
  {\phantomsubcaption\label{fig:anotherWow}}
  \caption{\panel{sub@fig:anotherAmazing}, This is also amazing.
    \panel{sub@fig:anotherWow}, Yes indeed.
  \label{fig:astoundingOne}}
\end{figure}

\clearpage

\begin{table}
\begin{center}
  \caption{\label{tab:description} Great table, isn't it?}
  \rowcolors{2}{white}{lightlightgray}
  \begin{tabular}{ llllc }
    \Xhline{2pt}
    My & great & table & wow & amazing\\
    \hline
    Wow & 2 & 4 & 3 & 2\\
    \Xhline{2pt}
  \end{tabular}
\end{center}
\end{table}

\clearpage

%% Reset numbering
\pagenumbering{arabic}

\section*{Supplementary Information}

\subsection*{Supplementary Notes}

\subsection*{\Cref{snote:great}: Great note}\snote{snote:great}

Really remarkable stuff here.

\qc{otherFixInSupplemental}{\highLight{We also did stuff here for the revision
    that will appear in the rebuttal, but in the supplemental section here, so
    note the \texttt{qps} in the rebuttal.}}

\subsection*{Supplementary Figures}

\clearpage

\begin{sfigure}[htbp]
  \centering
  \includepdf[fitpaper=true,pages=-]{supplementary/figure_LUAD_CRC.pdf}
\end{sfigure}

\clearpage

\begin{sfigure}[htbp]
  \centering
  {\phantomsubcaption\label{sfig:okay}}
  {\phantomsubcaption\label{sfig:maybe}}
  {\phantomsubcaption\label{sfig:alright}}
  \caption{\panel{sub@sfig:okay,sub@sfig:maybe}, This describes so many cool
    things. \panel{sub@sfig:okay}, This is okay. \panel{sub@sfig:maybe}, This is
    meh. \panel{sub@sfig:alright}, This is alright.}
  \label{sfig:itsokay}
\end{sfigure}

\clearpage

\section*{Supplementary Tables}

\captionof{stab}{\label{stab:goodStuff} Here is some info.}
